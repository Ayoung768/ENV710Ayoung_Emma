% Options for packages loaded elsewhere
\PassOptionsToPackage{unicode}{hyperref}
\PassOptionsToPackage{hyphens}{url}
%
\documentclass[
]{article}
\usepackage{amsmath,amssymb}
\usepackage{iftex}
\ifPDFTeX
  \usepackage[T1]{fontenc}
  \usepackage[utf8]{inputenc}
  \usepackage{textcomp} % provide euro and other symbols
\else % if luatex or xetex
  \usepackage{unicode-math} % this also loads fontspec
  \defaultfontfeatures{Scale=MatchLowercase}
  \defaultfontfeatures[\rmfamily]{Ligatures=TeX,Scale=1}
\fi
\usepackage{lmodern}
\ifPDFTeX\else
  % xetex/luatex font selection
\fi
% Use upquote if available, for straight quotes in verbatim environments
\IfFileExists{upquote.sty}{\usepackage{upquote}}{}
\IfFileExists{microtype.sty}{% use microtype if available
  \usepackage[]{microtype}
  \UseMicrotypeSet[protrusion]{basicmath} % disable protrusion for tt fonts
}{}
\makeatletter
\@ifundefined{KOMAClassName}{% if non-KOMA class
  \IfFileExists{parskip.sty}{%
    \usepackage{parskip}
  }{% else
    \setlength{\parindent}{0pt}
    \setlength{\parskip}{6pt plus 2pt minus 1pt}}
}{% if KOMA class
  \KOMAoptions{parskip=half}}
\makeatother
\usepackage{xcolor}
\usepackage[margin=1in]{geometry}
\usepackage{graphicx}
\makeatletter
\def\maxwidth{\ifdim\Gin@nat@width>\linewidth\linewidth\else\Gin@nat@width\fi}
\def\maxheight{\ifdim\Gin@nat@height>\textheight\textheight\else\Gin@nat@height\fi}
\makeatother
% Scale images if necessary, so that they will not overflow the page
% margins by default, and it is still possible to overwrite the defaults
% using explicit options in \includegraphics[width, height, ...]{}
\setkeys{Gin}{width=\maxwidth,height=\maxheight,keepaspectratio}
% Set default figure placement to htbp
\makeatletter
\def\fps@figure{htbp}
\makeatother
\setlength{\emergencystretch}{3em} % prevent overfull lines
\providecommand{\tightlist}{%
  \setlength{\itemsep}{0pt}\setlength{\parskip}{0pt}}
\setcounter{secnumdepth}{-\maxdimen} % remove section numbering
\ifLuaTeX
  \usepackage{selnolig}  % disable illegal ligatures
\fi
\IfFileExists{bookmark.sty}{\usepackage{bookmark}}{\usepackage{hyperref}}
\IfFileExists{xurl.sty}{\usepackage{xurl}}{} % add URL line breaks if available
\urlstyle{same}
\hypersetup{
  pdftitle={The Effects of Offshore Wind on Bottlenose Dolphin Strandings along the United States East Coast},
  pdfauthor={Emma Beyer \& Ayoung Kim},
  hidelinks,
  pdfcreator={LaTeX via pandoc}}

\title{The Effects of Offshore Wind on Bottlenose Dolphin Strandings
along the United States East Coast}
\author{Emma Beyer \& Ayoung Kim}
\date{2024-04-12}

\begin{document}
\maketitle

\hypertarget{introduction}{%
\subsection{Introduction}\label{introduction}}

Wind energy stands out as a renewable and sustainable energy source,
harnessing wind to generate power. Utilizing propeller-like blades, wind
turbines capture wind energy to spin a generator and produce electricity
(Office of Energy Efficiency \& Renewable Energy n.d.). Wind turbines
find application in three main methods: land-based, offshore, and
distributed wind (Office of Energy Efficiency \& Renewable Energy n.d.).
Among these, offshore wind turbines are situated in the ocean, tapping
into the robust winds that sweep across the sea. Thanks to higher wind
speeds and consistent directions, offshore wind turbines boast greater
efficiency compared to their onshore counterparts. Additionally, they
face lower risks of physical interference, setting them apart from other
forms of wind energy farms (National Grid n.d.).

One of the main concerns surrounding the installation and use of
offshore wind is the potential effects that it could have on marine
wildlife like cetaceans (whales, dolphins, and porpoise). Cetaceans
could be affected by offshore wind in two ways, noise and habitat
manipulation (Thompson et al 2010). In the paper by Thompson et al, they
found that noise from pile-driving could impact dolphin behavior up to
40 km from the site. This would be concerning because the increase in
noise could cause physical hearing damage to dolphins, disturb foraging
and social behavior, or forcing the dolphins to change their migration
patterns to avoid the area (Thompson et al 2010). There is also a
concern that these turbines would alter the habitat enough to affect the
dolphins' population. It's possible that these turbines could lead to
either habitat loss by deterring certain species or habitat creation
like the attraction of certain species to the structure similar to what
has been observed at oil platforms (Thompson et al 2010). This could
change the prey populations, in turn changing where the dolphins are
feeding. Additionally, the turbines could pose a physical danger to the
dolphins due to the increase in obstructions within the water column and
increased boat traffic around the turbines. Mitigation measures such as
strategic placement of wind farms, acoustic monitoring, and seasonal
construction restrictions are crucial to minimize these potential
effects and safeguard dolphin populations in offshore environments.

NOAA maintains the National Stranding Database, which is an open access
data set created by every reported marine mammal stranding within US
waters. A stranding is when a whale, dolphin, or porpoise is found dead,
either on the beach or floating in the water, or alive on the beach and
unable to be returned to the water (NOAA 2024). One of their largest
catalogs is the record of bottlenose dolphin (Tursiops truncatus)
strandings along the east coast of the United States. This happens to be
the same area where there are current offshore wind projects and where
there are many proposed projects. Due to this increase in proposed
offshore wind projects along the east coast, we are interested in
exploring how the presence of offshore wind farms has affected
bottlenose dolphin standings along the eastern United States. Within
this analysis we will be looking at how the length of each stranded
dolphin changes by states with and without offshore wind development,
what the odds are of finding different age classes in states with and
without offshore wind development, and what proportion of the stranded
dolphins in states with and without offshore wind development are male
or female.

\hypertarget{methodology}{%
\subsection{Methodology}\label{methodology}}

The data from the National Stranding Database is collected by members of
the National Stranding Network who respond to a marine mammal stranding.
The member will fill out a Level A Stranding Report that includes
information like species, age class, sex, date, location, length,
weight, and evidence of either boat collision, fisheries interactions,
or human interference to name a few. This dataset is quite extensive and
included strandings from 2017-2019 and contained 69 variables.

For the purposes of this analysis we only included the following
variables: year of observation, age class, sex, state of reported
stranding, and length of the dolphin. Within this analysis our
independent variables will be dolphin age class (Age.Class), dolphin sex
(Sex), states of reported strandings (State), and an additional binomial
variable of states that have active offshore wind projects (coded as a
1) and states that do not have active offshore wind projects (coded as
0) (turbine\_presence). The dependent variables within this analysis
will be the number of dolphin strandings and the length of each dolphin
(Length). Note that there was missing data within this dataset. There
were some states that did not record the weight of the stranded
dolphins, and some of the recorded stranding had missing or NA values.
To maintain an adequate number of observations we chose to focus on the
variable length over weight and cleaned the dataset by removing
observations with missing/NA values within our variables of interest.

Because the variables we aim to evaluate are binary and categorical, we
employed two logistic regression models using glm() and two linear
regression models throughout the analysis. We fitted fit\_1 and fit\_2
with logistic regression models utilizing turbine\_presence, a binary
variable, to discern the associations between turbines and bottlenose
dolphin strandings, incorporating their length. For fit\_3 and fit\_4,
we employed linear regressions to ascertain the difference in length
among states and the significance of dolphin length across states with
offshore wind farms. To facilitate a clearer analysis, we cleaned up the
datasets as described above and denoted them as `cleaned\_strandings'
and `turbine\_data'. To validate the assumptions of the models, we
examined the intercepts' p-values and generated qq-plots for some
regression models to assess how well the results support the
assumptions.

\hypertarget{results}{%
\subsection{Results}\label{results}}

After cleaning the data to remove all the blank, unknown, and NA values
there were a total of 1419 recorded strandings from 2017-2019 along the
east coast. 80 of these strandings were recorded in states with active
offshore wind development. Overall, 15 states were included within this
dataset, and only 4 states had active offshore wind development during
our study period. These states were New York, Massachusetts, Rhode
Island, and Virginia. When looking at the number of stranding per state,
Texas (306) and Florida (295) had the most strandings while Maine (1)
and Rhode Island (1) had the least. When looking at the length of
stranded dolphins, the mean length across all 15 states was 194.2091
(cm) and the variance was 65.37 (cm) while the mean length across the 4
offshore wind states was 198.175 (cm) and the variance was 78.36 (cm).
When looking at the age class of each stranded dolphin, the mean length
for each age class was PUP/CALF 107.1709 cm , YEARLING 163.3439 cm,
SUBADULT 194.0060 cm, and ADULT 242.8233 cm. The number of strandings
per age class for all states were ADULT 669, PUP/CALF 335, SUBADULT 308,
and YEARLING 107, while the number of strandings per age class for
offshore wind states were ADULT 32, PUP/CALF 21, SUBADULT 26, and
YEARLING 1. When looking at the strandings by dolphin sex, the mean
length for females was 191.8816 cm and 195.8511 cm for males. Within the
whole dataset there were 587 stranded females and 832 stranded males,
while in the states with offshore wind there were 32 stranded females
and 48 stranded males.

\#\#\#Fit 1

First of all, to assess the probability of turbine presence, we fitted
an intercept-only model, Fit\_1, where turbine\_presence = β0. The
estimated value of β0 is -2.8177, which represents the log odds of
bottlenose strandings in the states with offshore wind turbines.
According to the summary, the predicted probability of finding the
stranded bottlenose dolphin in the states with offshore wind farms in
the east coast area would be exp(-2.8177)=0.0597, approximately 5.97\%.

\#\#\#Fit 2

\[H0: There is no difference in offshore wind turbine presence across different age classes. \]
\[Ha: There is a difference in offshore wind turbine presences across different age classes. \]

We hypothesized that there is no difference in offshore wind turbine
presence across different age classes to evaluate how the presence of
wind turbines affected strandings across four different age groups:
Adult, Yearling, Subadult, and Pup/calf. The estimate of B0 is -2.99,
representing the log odds of finding offshore wind turbines at the
locations where the adult bottlenose dolphin strandings were found. The
prob = exp(0.05). Compared to Fit\_1, the estimated value slightly
decreased. The reference is adult strandings, and B1, B2, and B3
represent Pup/calf, Subadult, and Yearling, respectively. Each of these
estimates represents an incremental adjustment to the intercept.

The predicted probability of finding offshore wind turbines at locations
where adult strandings are found is exp(-2.9910), which equals 0.05, or
around 5\%. The predicted probability of Pup/Calf is exp(-2.9910+0.2862)
= 0.067, approximately 6.7\%. For the subadult strandings, the predicted
probability would be exp(-2.9910+0.6072) = 0.092, around 9.2\%. Lastly,
for the yearling population, the predicted probability is
exp(-2.9910-1.6724) = 0.0094, approximately 0.94\%.

We found that the predicted probability of pup/calf and subadult
strandings were higher than that of the reference age group (Adult),
while it declined for the yearling stranding populations. Considering
that the p-value of the reference group is \textless2e-16, we could
conclude that it is statistically significant with a high level of
confidence.

Also, using the QQplots, we assessed how well the model fits the data
and the assumptions. We used par(mfrow=c(2,2)) to display 4 plots,
Residuals vs Fitted, Q-Q Residuals, Scale-Location, and Residuals vs
Leverage plots. The Residuals vs Fitted graph presents that the data is
linear and homoscedasticity for the most part. In the Q-Q Residuals
plot, the two groups of clustered points were found. More than half of
the points followed the line, but from 2.0 theoretical quantiles, the
points were not clustered along the line. Also the plot shows the 2
representative points outside of normality (645, 1116). The
Scale-Location plot with a gentle line presents that there is no
triangular shape to the point and therefore the data is homoscedastic.
The Residual vs Leverage plot suggests that the points (720, 1026, 1116)
have the biggest effects of the parameter estimates. These results
suggest that the model regression lines reasonably fit the data.

\#\#\#Fit 3

We conducted an additional regression model matching the length of
bottlenose strandings and the states to test the hypothesis below,
evaluating how the lengths differ across states using the
cleaned\_strandings data.

\[H0: There is no difference in the length of bottlenose dolphin strandings across the states.\]
\[Ha: There is a difference in the length of bottlenose dolphin strandings across the states.\]

The reference for the model is AL, with an estimated value of 188.747.
This indicates the estimated mean length of the bottlenose dolphin
strandings in AL. Compared to AL, DE, NJ, TX, and VA have decreases in
length of 16.251, 109.191, 8.427, and 4.075, respectively, while FL, GA,
LA, MA, MD, ME, MS, NC, NY, RI, SC, and TX show increases in length of
15.847. Based on the results, we found that ME and RI had significantly
higher values in length compared to AL, while NJ had relatively lower
values in length.

\#\#\#Fit 4

For this model the value of \(\beta\)\textsubscript{0} is 260.89 cm.
This is the expected length of stranded dolphins in MA. The value of
\(VA\beta\)\textsubscript{1} is -76.22 cm. This is the difference in
expected length between MA and VA. \(NY\beta\)\textsubscript{2} is
-59.98 cm. This is the difference in expected length between MA and NY.
The value of \(RI\beta\)\textsubscript{3} is 42.11 cm. This is the
difference in expected length between MA and RI.

\[H_0: No difference in length between the states with offshore wind \]
\[H_a: There is a difference in length between the states with offshore wind\]

There was a significant difference in length of dolphins found in
Massachusetts (\(R^2\)=0.08104, \(p\)=2.62e\(^{-16}\)) and in Virginia
(\(R^2\)=0.08104, \(p\)=0.00619\$) . Therefore, we can reject the null
hypothesis that there is no difference in length of stranded dolphins in
states with offshore wind.

The Residuals vs Fitted graph suggests that the data is linear and
homoscedasticity for the most part. The Q-Q Residuals plot suggests that
there are three points outside of normality (1702, 1798, 1809), but a
majority of the points follow the normality line. The Scale-Location
plot suggests that there is no triangular shape to the point and
therefore the data is homoscedastic. The Residual vs Leverage plot
suggests that the points (1702, 1798, 1116) have the biggest effects of
the parameter estimates. These results suggest that the regression lines
reasonably fit the data.

\hypertarget{discussionresults}{%
\subsection{Discussion/Results}\label{discussionresults}}

There are only four states that have active offshore wind projects along
the east coast, Virginia, Rhode Island, New York, and Massachusetts. We
found that the states with the most strandings were Texas and Florida
(more southern states), while the states with the least amount were
Maine and Rhode Island (more northern states).

The lengths of stranded dolphins across all states (194.2091 cm) and
offshore wind states (198.175 cm) were similar. When looking at the
length of the dolphins across different age groups, we found that the
length tended to increase with each progressive age group. Our results
were consistent between all states and offshore wind states where the
age group with the most strandings were Adults and the age group with
the least strandings were Yearlings. We can conclude from Fit 2 that the
adults age group was the most vulnerable age group, since the number of
adults had the most strandings overall (Figure 1) and this significance
was confirmed by the low p-value of \textless2e-16. We compared the
lengths of bottlenose dolphins found in 15 different states, both with
and without offshore wind turbines. Our findings indicate that Maine and
Rhode Island exhibit significantly higher values, while New Jersey shows
a lower value in length. We found that there was a significant
difference in the lengths of stranded dolphins in both Virginia (which
were the smallest) and Massaschsetts (which was the largest other than
Rhode Island) when compared to the other offshore wind states (Figure
6).

This population was isolated to the east coast of the United States and
only took into account the distribution of dead/stranded bottlenose
dolphins. From this data, we can only infer on where the dolphins were
dying and not on where the true living populations were living. This
data was collected between 2017 and 2019, so we were only able to infer
about the dolphin population within this time frame.

We could not find previous studies analyzing the dead strandings of
bottlenose dolphins to examine how offshore wind turbines in the ocean
affected them. While some earlier research has explored the associations
between the installation of wind energy farms and marine ecosystems,
focusing on marine mammals such as whales, the specific impact on
bottlenose dolphin populations remains unexplored. Given the difficulty
of tracking live marine mammal populations, investigating dolphin
strandings provided an alternative approach to understanding the
relationship between offshore wind turbines and bottlenose dolphin
populations.

Drawing from studies on whales, we anticipated negative impacts of
offshore wind turbines on bottlenose dolphin populations, attributing
these to factors such as noise pollution and alterations in the
ecosystem. Additionally, we hypothesized that there would be no
significant difference in vulnerability across age groups; however,
contrary to our expectations, we found that the adult group was more
vulnerable than younger age groups.

One of the limitations of our study is the abundance of missing data
marked as NA and inconsistencies in measurement methods. Observations
from areas along the east coast with offshore wind turbines were fewer
than anticipated, and there was a lack of consistent weight measurement,
complicating the inclusion of weight as a variable in our regression
models. To address these limitations, we had to clean up the datasets,
resulting in fewer observations overall.

\hypertarget{description-of-participant-roles}{%
\subsection{Description of Participant
Roles}\label{description-of-participant-roles}}

For this analysis, both participants worked together to find and compile
the datasets and worked together to select the necessary variables and
models that best fit our data. All models and results were reviewed,
discussed, and edited together. Ayoung worked on the wind energy section
of the introduction, the analysis overview section in the methodology,
and the statistical analysis/results of Fits 1/2/3. Emma worked on the
bottlenose dolphin strandings section and the explanation of the dataset
in the introduction, the description of the data within the methodology,
the descriptive statistics of the dataset, and the statistical
analysis/results of Fit 4.

\hypertarget{bibliography}{%
\subsection{Bibliography}\label{bibliography}}

Thompson, P. M., D. Lusseau, t. Barton, D. Simmons, J. Rusin, and H.
Bailey. 2010. Assessing the responses of coastal cetaceans to the
construction of offshore wind turbines. Marine Pollution Bulletin
60:1200-1208.

NOAA Fisheries. 2024. Understanding Marine Wildlife Stranding and
Response.
\url{https://www.fisheries.noaa.gov/insight/understanding-marine-wildlife-stranding-and-response}
(04/01/2024)

NOAA Fisheries. 2024. National Stranding Database Public Access.
\url{https://www.fisheries.noaa.gov/national/marine-life-distress/national-stranding-database-public-access}
(04/01/2024)

Office of Energy Efficiency \& Renewable Energy. 2024. How Do Wind
Turbines Work?
\url{https://www.energy.gov/eere/wind/how-do-wind-turbines-work}
(04/04/2024)

National Grid. 2024. Onshore vs Offshore Wind Energy: What's the
Difference? Onshore vs offshore wind energy: what's the difference?
\textbar{} National Grid Group (04/04/2024)

\hypertarget{appendix}{%
\subsection{Appendix}\label{appendix}}

\begin{figure}

{\centering \includegraphics[width=0.7\linewidth]{BeyerEKimAFinalPaper_files/figure-latex/plot 1-1} 

}

\caption{\label{fig:figs} The number of stranded dolphins reported in each state across the East Coast. States with active offshore wind projects are labeled in red.}\label{fig:plot 1}
\end{figure}

\begin{figure}

{\centering \includegraphics[width=0.7\linewidth]{BeyerEKimAFinalPaper_files/figure-latex/plot 2-1} 

}

\caption{\label{fig:figs} The distribution of length (cm) of stranded dolphins reported in each state across the East Coast. States with active offshore wind projects are labeled in red.}\label{fig:plot 2}
\end{figure}

\begin{figure}

{\centering \includegraphics[width=0.7\linewidth]{BeyerEKimAFinalPaper_files/figure-latex/plot 3-1} 

}

\caption{\label{fig:figs} The number of strandings for each age class of bottlenose dolphin and whether or not there is a presence of offshore wind.}\label{fig:plot 3}
\end{figure}

\begin{figure}

{\centering \includegraphics[width=0.7\linewidth]{BeyerEKimAFinalPaper_files/figure-latex/plot 4-1} 

}

\caption{\label{fig:figs} The number of strandings for each sex and whether or not there is a presence of offshore wind.}\label{fig:plot 4}
\end{figure}

\begin{figure}

{\centering \includegraphics[width=0.7\linewidth]{BeyerEKimAFinalPaper_files/figure-latex/plot 5-1} 

}

\caption{\label{fig:figs} The distribution of lengths (cm) of stranded bottlenose dolphins across all states.}\label{fig:plot 5}
\end{figure}

\begin{figure}

{\centering \includegraphics[width=0.7\linewidth]{BeyerEKimAFinalPaper_files/figure-latex/plot 6-1} 

}

\caption{\label{fig:figs} Linear regression of the lengths (cm) of stranded dolphin across states with offshore wind projects.}\label{fig:plot 6}
\end{figure}

\end{document}
